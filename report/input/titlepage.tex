\thispagestyle{empty}
\begin{center}

    \definecolor{ul-red}{RGB}{220,29,39}
    \begin{figure}[htb!]
        \centering
        \includegraphics[width=0.09\linewidth]{ul-logo}
    \end{figure}
    \LARGE{\textsc{University of Ljubljana}}\\
    \Large{\textsc{Faculty of {\color{ul-red} mathematics and physics}}}\\[1ex]
    \large{\textsc{Department of physics}}\\
    \vspace{5ex}
    \huge{Data Acquisition and Processing}\\
    \rule{0.9\textwidth}{0.2pt}\\[1ex] \LARGE{\textbf{Final report: Designing and implementing a real-time Hilbert transformer}}
    \rule{0.9\textwidth}{0.2pt}

    \vspace{1ex}

    \begin{minipage}[t]{0.80\textwidth}
        \normalsize{\textsc{Author:}} \hfill \large{\textsc{Student ID}:}\\
    \large{Elijan Jakob Mastnak} \hfill \large{28181157}
    \end{minipage}

\end{center}

\vspace{5ex}
\begin{center}
    \textbf{Assignment}\\[2mm]
    \begin{minipage}[t]{0.80\textwidth}
        Use the window method with an appropriate window function to design an FIR band-pass Hilbert transform filter for use with signals sampled at $ f_{\mathrm{s}} = \SI{44100}{\hertz} $.
        The filter passband should span the frequency range from \SI{1000}{\hertz} to \SI{2000}{\hertz}; in the passband, the input and output signals should differ in phase by $ \pi/2 $ (as for a Hilbert transformer) and in amplitude by no more than one \SI{1}{\decibell}.
        For frequencies below \SI{500}{\hertz} and above \SI{2500}{\hertz}, the output signal's amplitude should be at least \SI{40}{\decibell} less than the input amplitude.
        Compute the FIR filter's coefficients, test the filter in an offline simulation, and finally ensure the filter works in real time.
        
    \end{minipage}

    \vfill
    \large{Ljubljana, September 2021}
\end{center}
